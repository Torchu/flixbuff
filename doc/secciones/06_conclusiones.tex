\chapter{Conclusiones y trabajos futuros}
\section{Conclusiones}
En este capítulo vamos a analizar el grado de cumplimiento de los objetivos definidos en el
\hyperref[sec:objetivo]{primer capítulo} y plantear las posibles lineas futuras de trabajo.\\

Los dos principales desafíos de este proyecto eran permitir a los usuarios crear reseñas de la serie que quieran y
permitirles seguir a otros usuarios para tener un rápido acceso a las reseñas de éstos. Ambos objetivos se han
conseguido en la aplicación web tal y como se refleja en la \hyperref[chap:implementación]{implementación}.\\

Esta nueva aplicación consigue con éxito satisfacer las necesidades \textit{seriéfilas} y sociales de los usuarios, que
se encontraban huérfanos de una plataforma en la que expresar su opiniones y leer las de los demás sobre sus series
favoritas.\\

En lo personal, este proyecto me ha servido para aprender y vivir la experiencia de realizar un proyecto de ingeniería
desde cero. Plantear los objetivos, estimar los costes y establecer y seguir una metodología de trabajo ágil, son
conocimientos que me aportarán mucho más valor como ingeniero en el campo profesional.

\section{Trabajos futuros}
En esta sección pretendo exponer posibles lineas de trabajo a seguir para sacar más valor a la solución encontrada:

\begin{itemize}
    \item Permitir a los usuarios personalizar su perfil, mejorar la UI/UX del cliente o dar soporte a aplicaciones
    móviles enriquecerían la experiencia de los usuarios, mejorando su relación con la aplicación y atrayendo a más
    gente a su uso.
    \item Dar soporte a películas permitiría a los usuarios tener todas sus reseñas en una misma aplicación.
    \item Ofrecer a empresas un servicio de pago de análisis de datos que les permita conocer los gustos de ciertos
    sectores de la población. De esta forma, se podrían financiar los costes de la aplicación sin necesidad de anuncios
    o cobro a los usuarios.
\end{itemize}
