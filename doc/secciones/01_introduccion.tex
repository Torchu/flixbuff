\chapter{Introducción}

Este proyecto es software libre, y está liberado con la licencia \cite{gplv3}.

\section{Motivación}
Con la irrupción y popularidad de diferentes servicios de streaming como Netflix o HBO, el catálogo de series que uno puede ver ya no se limita a las que echen por la televisión. Ante este vasto mar de opciones, surgen dos problemas:
\begin{enumerate}
    \item Con tantas opciones, no somos capaces de discernir las que nos podrían interesar, causando el efecto contrario, que nos pasemos horas buscando para al final no ver nada.
    \item Como seres sociales que somos, nos surge la imperiosa necesidad de saber qué series se han visto nuestros conocidos y conocer su opinión sobre las mismas, a la vez que contarles a ellos nuestra opinión sobre las que nos hemos visto.
\end{enumerate}

El primer problema es resuelto en cierta forma por webs que clasifican y/o puntúan las series del catálogo actual. El segundo, a día de hoy sigue sin ser resuelto y está más que demostrada su necesidad por los usuarios de Twitter, en los innumerables hilos en los que exponen su opinión sobre las series que han visto a lo largo del año.
\begin{figure}[]
	\centering	
	\includegraphics[scale=0.25]{img/twitter-thread-1.png}
    \includegraphics[scale=0.253]{img/twitter-thread-2.png}
	\caption{ @AcolyteJoel y @IniciatiMarvel, usuarios de Twitter, comentando su opinión sobre las series que han visto. }
    \label{fig:twitter_threads}
\end{figure}
