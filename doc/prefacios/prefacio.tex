\thispagestyle{empty}

\begin{center}
{\large\bfseries Flixbuff \\ Una red social para seriéfilos }\\
\end{center}
\begin{center}
Víctor Cabrita Gómez\\
\end{center}

\vspace{0.7cm}

\noindent\textbf{Resumen}\\

Con la irrupción y popularidad de diferentes servicios de streaming como Netflix, HBO, etc. el catálogo de series que uno puede ver ya no se limita a las que echen por la televisión. En este vasto mar de opciones, todos tenemos la misma pregunta: ``¿Qué debería ver?''. A pesar de existir infinidad de sitios web que clasifican las series de moda en un ranking, lo que a uno más le interesa es la opinión de su círculo cercano.\\

Flixbuff es la aplicación ideal para aquellos que quieren saber qué opinan sus amigos sobre las series del momento. Es una red social en la que los usuarios pueden escribir su opinión sobre las series que hayan visto y leer la opinión de los usuarios que siguen sobre las mismas.

\cleardoublepage{}

\begin{center}
	{\large\bfseries Flixbuff \\ A social network for TV Series enthusiasts}\\
\end{center}
\begin{center}
	Víctor Cabrita Gómez\\
\end{center}
\vspace{0.7cm}

\noindent\textbf{Abstract}\\

With the increase in popularity of different streaming services such as Netflix, HBO, etc. the catalog of TV series that one can see already no longer limits itself to those that broadcast on television. In this vast market of options, everyone has the same question: ``What should I watch?'' Despite the fact that there are many websites that classify TV series in a ranking, what one more interested is the opinion of their friends.\\

Flixbuff is the ideal application for those who want to know what their friends think about the TV series of the moment. It is a social network in which users can write their opinion about the TV series they have seen and read the opinion of users that they follow.

\cleardoublepage{}

\thispagestyle{empty}

\noindent\rule[-1ex]{\textwidth}{2pt}\\[4.5ex]

D. \textbf{Juan Julián Merelo Guervós}, profesor del departamento de Arquitectura y Tecnología de Computadores.

\vspace{0.5cm}

\textbf{Informo:}

\vspace{0.5cm}

Que el presente trabajo, titulado \textit{\textbf{Flixbuff}},
ha sido realizado bajo mi supervisión por \textbf{Víctor Cabrita Gómez}, y autorizo la defensa de dicho trabajo ante el tribunal
que corresponda.

\vspace{0.5cm}

Y para que conste, expiden y firman el presente informe en Granada a Septiembre de 2022.

\vspace{1cm}

\textbf{El/la director(a)/es: }

\vspace{5cm}

\noindent \textbf{Juan Julián Merelo Guervós}

\chapter*{Agradecimientos}

A mis padres por darme la posibilidad de comenzar este viaje, a mis amigos, \textit{Los Chavales del Barrio}, por todo su apoyo y a todas las personas que me han acompañado en este viaje.