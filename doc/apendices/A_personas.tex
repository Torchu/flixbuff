\chapter{Personas}\label{chap:personas}
\section{Clara}

\begin{itemize}
  \item \textbf{Nombre y apellidos: } Clara Sánchez.
  \item \textbf{Nació en: } Reinosa.
  \item \textbf{Reside en: } Santander.
  \item \textbf{Edad: } 19.
  \item \textbf{Estado civil: } Soltera.
  \item \textbf{Estudia: } Magisterio infantil.
  \item \textbf{Rasgos de personalidad: } 
  \begin{itemize}
    \item Divertida.
    \item Carismática.
    \item Social.
  \end{itemize}
  \item \textbf{¿Cuáles son sus entornos?: } 
  \begin{itemize}
    \item Compañeros de clase y de la residencia de estudiantes.
    \item Amigos del pueblo.
    \item Su familia.
  \end{itemize}
  \item \textbf{¿Qué influencia su opinión?: } 
  \begin{itemize}
    \item Lo que piensan sus amigos.
    \item Lo que lee en las redes sociales.
  \end{itemize}
  \item \textbf{¿Cuál es su relación con la tecnología?} 
  \begin{itemize}
    \item Utiliza aplicaciones de ofimática con el ordenador para los trabajos de la universidad.
    \item Principalmente utiliza su dispositivo móvil y una tablet para los momentos de ocio.
  \end{itemize}
\end{itemize}

\section{Carlos}
\begin{itemize}
    \item \textbf{Nombre y apellidos: } Carlos Martín.
    \item \textbf{Nació en: } Santander.
    \item \textbf{Reside en: } Madrid.
    \item \textbf{Edad: } 27.
    \item \textbf{Estado civil: } Soltero.
    \item \textbf{Estudió: } Comunicación audiovisual.
    \item \textbf{Trabaja: } Ayudante de realizador en una cadena de televisión.
    \item \textbf{Rasgos de personalidad: } 
    \begin{itemize}
      \item Tranquilo.
      \item Culto.
      \item Serio.
    \end{itemize}
    \item \textbf{¿Cuáles son sus entornos?: } 
    \begin{itemize}
      \item Compañeros de trabajo.
      \item Amigos de su ciudad natal.
      \item Otros escritores amateurs.
    \end{itemize}
    \item \textbf{¿Qué influencia su opinión?: } 
    \begin{itemize}
      \item Las opiniones de gente que respeta.
      \item Lo que lee en revistas especializadas.
    \end{itemize}
    \item \textbf{¿Cuál es su relación con la tecnología?} 
  \begin{itemize}
    \item Está acostumbrado a usar el ordenador para trabajar.
    \item Suele usar su portátil para su afición de escritor. Se siente más cómodo que con el teléfono móvil.
  \end{itemize}
  \end{itemize}